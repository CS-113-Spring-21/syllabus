\documentclass[a4paper]{article}

\usepackage{geometry}
\usepackage{hyperref}
\usepackage{tabularx}

% Roman section numbers
% \renewcommand{\thesection}{\Roman{section}} 
% \renewcommand\thesubsection{\Alph{subsection}}

\newcolumntype{C}{>{\centering\arraybackslash}X}

\begin{document}
\begin{center}
  {\bfseries {\huge CS 113 Discrete Mathematics}\\\bigskip
    {\large 3+0 cedits, Spring 2021, Habib University}}\\\medskip

\textit{``Computer science is no more about computers than astronomy is about telescope.''\\
  –- Edsger Dijkstra\\\medskip
  ``All models are wrong, but some are useful.'' -– George Box}
\end{center}
\medskip

\noindent
\begin{tabularx}{1.0\linewidth}{lX}
  Instructors:
  & \href{https://habib.edu.pk/SSE/dr-waqar-saleem/}{Waqar Saleem}, \href{https://habib.edu.pk/SSE/rameez-ragheb/}{Rameez Ragheb}, \href{https://habib.edu.pk/SSE/dr-shahid-hussain/}{Shahid Hussain}\\
  TAs:: & See Live Syllabus \\
  Course sites: & \href{https://hulms.instructure.com/courses/1262}{LMS (Canvas)}, \href{https://habibedu.workplace.com/groups/219075873051099}{Workplace}, \href{https://bit.ly/3nQPsSR}{Live Syllabus}\\
  Course Prerequisites: &  \textit{None}\\
  Software Prerequisites: &  \LaTeX\\
  Hardware Prerequisites: & Computer with mic, camera, and Internet connection\\
  Content Area: & This course is part of CS Foundation. It is required for both a major and minor in CS. For other students, it can be counted as either a Free Elective, University Elective, SSE Elective, CS Elective, or CS requirement.\\
Campus Safety Policy: & Please read the current campus safety policy and protocols if the classes are in-person.  
\end{tabularx}
\medskip

\section{Rationale}

Computer Science is the study of computation performed inevitably on discrete machines.  Modeling and analysis of these computations require formal methods to reason about them as well as a mathematics that deals with discrete events and entities.  In addition, computation follows a certain logic. Understanding this logic helps to not only understand computation but to prove properties of algorithms like their complexity and correctness.

This course is primarily an exercise in proofs. After covering some of the logic required to perform proofs, we study various discrete entities and techniques which are common in the design and analysis of algorithms.  When studying these entities, our aim is to gain sufficient familiarity with them so as to prove their various properties. As such, this course equips students with essential mathematical tools that they will encounter in future Computer Science courses.  It develops a capacity for formal mathematical manipulation and abstract thought, both of which are essential for the successful pursuit of Computer Science.

\section{Course Aims and Outcomes}

\subsection{Course Objectives}

This course aims to:
\begin{itemize}
\item develop a capacity for reading and writing mathematical proofs,
\item introduce various discrete structures and techniques,
\item emphasize precision in thought and in writing through rigorous mathematical notation, and
\item develop a capacity for abstract thought necessary for the study of Computer Science.
\end{itemize}

\subsection{Course Learning Outcomes (CLOs)}

On successful completion of this course, you will:
\begin{enumerate}
\item correctly solve problems related to propositional and predicate logic,
\item correctly solve problems related to sets, relations, and functions,
\item correctly model suitable problems using graphs and combinatorics and apply known theorems on the models,
\item correctly apply proof techniques to prove mathematical statements, and
\item fruitfully collaborate on the solution, research, and presentation of problems, proofs, and techniques related to the above topics.
\end{enumerate}

\section{Program Learning Outcomes (PLOs) }

Coming soon.

\section{CLO to PLO mapping}

Coming soon.

% \noindent
% \begin{tabularx}{1.0\linewidth}{|X||*5{X|}}
%   \hline
%   &CLO 1& CLO 2& CLO 3& CLO 4& CLO 5\\  \hline\hline
%   PLO 1 & 25\%& 25\% & 25\% & 25\% & 25\%\\\hline
% \end{tabularx}

% Note – if the SLOs/CLOs are SMART then 1 SLO/CLO may target only one PLO, however, different SLOs/CLOs might contribute to a PLO.]

\section{Format and Procedures}


This is a highly theoretical course. You are encouraged to be attentive in lectures and do the assignments and readings in a timely manner. The instructors will make all efforts to closely follow the course textbook so that you have a ready reference. 

This is a 3 credit hour course. The rule of thumb for out-of-class time for a course is at least 2 hours of work outside class for every credit hour. In the previous iteration of this course, most students reported spending between 4 and 9 hours per week outside class. This may vary based on your comfort with mathematics and capacity to absorb and apply new ideas.

You must attend your weekly recitation which will provide deeper insight into the topics covered that week through practice problems.

\begin{description}
\item[Medium of Instruction] The course will be a blend of asynchronous and synchronous content for different sections. Asynchronous content will be shared before the start of the synchronous sessions and students are expected to come well prepared for the synchronous sessions after watching all uploaded asynchronous content.
\item[Time Journal] You are encouraged to maintain a journal to record the time that you spend on this course. This includes the time you spend watching asynchronous videos, attending live sessions, doing any background work, attempting the homework, filling the weekly feedback form, completing any other required forms, and so on. In short, any activity that you perform related to this course. 
\item[Consultation] Please utilize student hours and Ehsas hours of the instructors and TAs in order to discuss course related matters and queries.
\item[Course Material] All course resources (asynchronous videos, recorded sessions, reference books, articles and all other support material) will be made available through the course site on Canvas.
\item[Viva] Submission of homework assignments is in groups and may be followed up by a group viva called by the instructor. A viva will be called as necessary and need not apply to all groups or all assignments.
\item[Punctuality] Please respect deadlines. Submit your work by the indicated time. Incomplete work will receive partial credit. Late work will not be accepted or graded. 
\item[Contesting marks] Concerns regarding a score will be entertained by the respective instructor up to a week after the release of the score. Concerns raised later will not be entertained. 
\item[Grace marks] Requests for grace marks for whatever reason will not be entertained and each such request will result in a penalty of 1\% from the overall score. 
\item[Behavior] You are expected to maintain a behavior befitting Yohsin and acknowledging the classroom as a place of learning, exploration, and experimentation. The University’s standard policies on attendance, inclusivity, office hours, and academic integrity apply in this course. These are described below. 
\end{description}

\subsection{Engagement, Net-etiquette and Participation}
  \begin{description}
  \item[People, not boxes] In order to maintain a healthy class environment, you are encouraged to keep your camera on during live sessions. In case of extenuating circumstances that prevent you from doing so, you must communicate them to me beforehand over Canvas.
  \item[Names] Please make sure that your name on Zoom is the same as it appears on Canvas.
  \item[Communication] All official course communication will take place over LMS. It is your responsibility to stay up to date with it. 
  \end{description}

\section{Course Requirements}

This course requires comfort with mathematics. You will encounter topics that are new and challenging and will be required to absorb and apply them. You may also encounter some familiar topics but their treatment in this course may be new to you.

You are also required to have the capacity to self learn details of related tools, i.e. \LaTeX. You are highly encouraged to utilize the consultation hours of the course staff for any course discussion.
\medskip

\noindent\textbf{Course textbook}
\begin{itemize}
\item \textit{Discrete Mathematics and Its Applications (7th edition)}, by Kenneth H. Rosen. 
\end{itemize}

\noindent\textbf{Recommended supplementary texts}
\begin{itemize}
\item \textit{Mathematics for Computer Science, by Eric Lehman}, F Thomson Leighton, and Albert R Meyer. 
\item \textit{Discrete Math for Computer Science Students}, by Ken Bogart, Scot Drysdale, and Cliff Stein. 
\item \textit{Discrete and Combinatorial Mathematics: An Applied Introduction}, by Ralph Grimaldi. 
\item \textit{Concrete Mathematics: A Foundation for Computer Science}, by Ronald Graham, Donald Knuth, and Oren Patashnik.
\end{itemize}

\section{Assessments, SEL and Grading Procedures}

\subsection{Assessments and SEL}

Please see the Live Syllabus for up-to-date information on Assessment and SEL.

\subsection{Mapping of Assessments to Course Learning Outcomes}

This course’s assessment instruments are homework assignments (4), embedded quizzes (many), and a project. This is how they relate to the Course Learning Outcomes.
\smallskip

\noindent
\begin{tabularx}{1.0\linewidth}{|l||*5{C|}}
  \hline
  &CLO 1& CLO 2& CLO 3& CLO 4& CLO 5\\  \hline\hline
  Assignment 1 & & 1 & & & 1 \\\hline
  Assignment 2 & 1 &  & & 1 & 1 \\\hline
  Assignment 3 & & 1 & & 1 & 1 \\\hline
  Assignment 4 & & & 1 & 1 & 1 \\\hline
  Quizzes & 1 & 1 & 1 & 1 & \\\hline
  Project & & & & & 1 \\\hline
\end{tabularx}

\subsection{Grading Scale}

The course will follow the standard DSSE grading scale given below. Please consult the Live Syllabus for details of the course assessments.

\begin{center}
  \begin{tabular}{|*3{c|}}
\hline
Letter Grade & Numerical Equivalent & Percentage\\\hline\hline
A+ & 4.00 & [95-100] \\\hline
A & 4.00 & [90-95) \\\hline
A- & 3.67 & [85-90) \\\hline
B+ & 3.33 & [80-85) \\\hline
B & 3.00 & [75-80) \\\hline
B- & 2.67 & [70-75) \\\hline
C+ & 2.33 & [67-70) \\\hline
C & 2.00 & [63-67) \\\hline
C- & 1.67 & [60-63) \\\hline
F & 0.00 & [0-60) \\\hline
  \end{tabular}
\end{center}

\section{Attendance Policy}

Students are expected to watch all the asynchronous videos and attend the synchronous classes. Attendance will be marked. However, a student may notify their instructor on Canvas beforehand or up to an hour later if they miss a class. Otherwise, they will be marked absent. Attendance contributes to SEL and the related policy is outlined in the Live Syllabus.

I reserve the right to have you dropped from the course if you consistently maintain a low attendance, which will reflect in a low SEL score. You will receive one warning beforehand.

This policy will be updated once classes resume in person.

\section{Accommodations for students with disabilities}

In compliance with the Habib University policy and equal access laws, I am available to discuss appropriate academic accommodations that may be required for student with disabilities. Requests for academic accommodations are to be made during the first two weeks of the semester, except for unusual circumstances, so arrangements can be made. Students are encouraged to register with the Office of Academic Performance to verify their eligibility for appropriate accommodations.

\section{Inclusivity Statement}


We understand that our members represent a rich variety of backgrounds and perspectives. Habib University is committed to providing an atmosphere for learning that respects diversity. While working together to build this community we ask all members to:
\begin{itemize}
\item share their unique experiences, values and beliefs
\item be open to the views of others 
\item honor the uniqueness of their colleagues
\item appreciate the opportunity that we have to learn from each other in this community
\item value each other’s opinions and communicate in a respectful manner
\item keep confidential discussions that the community has of a personal (or professional) nature 
\item use this opportunity together to discuss ways in which we can create an inclusive environment in this course and across the Habib community 
\end{itemize}

%%%%%%%%%%%%%%%%%%%%%%%%%%%%%%%%%%%%%%%%%%%%%%%%%%

\section{Student hours}

Please consult the Live Syllabus for up-to-date information on the student hours of the course staff. 

\section{Academic Integrity}

Each student in this course is expected to abide by the Habib University Student Honor Code of Academic Integrity. Any work submitted by a student in this course for academic credit will be the student's own work. 

For this course, collaboration is allowed with your buddy in the following instances: \textbf{homework assignments and project}.

Scholastic dishonesty shall be considered a serious violation of these rules and regulations and is subject to strict disciplinary action as prescribed by Habib University regulations and policies. Scholastic dishonesty includes, but is not limited to, cheating on exams, plagiarism on assignments, and collusion. 

\paragraph{PLAGIARISM} Plagiarism is the act of taking the work created by another person or entity and presenting it as one’s own for the purpose of personal gain or of obtaining academic credit. As per University policy, plagiarism includes the submission of or incorporation of the work of others without acknowledging its provenance or giving due credit according to established academic practices. This includes the submission of material that has been appropriated, bought, received as a gift, downloaded, or obtained by any other means. Students must not, unless they have been granted permission from all faculty members concerned, submit the same assignment or project for academic credit for different courses. 

\paragraph{CHEATING} The term cheating shall refer to the use of or obtaining of unauthorized information in order to obtain personal benefit or academic credit. 

\paragraph{COLLUSION} Collusion is the act of providing unauthorized assistance to one or more person or of not taking the appropriate precautions against doing so.

All violations of academic integrity will also be immediately reported to the Student Conduct Office.  

You are encouraged to study together and to discuss information and concepts covered in the lecture with other students. You can give ``consulting'' help to or receive ``consulting'' help from such students. However, this permissible cooperation should never involve one student having possession of a copy of all or part of work done by someone else, in the form of an e-mail, an e-mail attachment file, a diskette, or a hard copy. 

Should copying occur, the student who copied work from another student and the student who gave material to be copied will both be in violation of the Student Code of Conduct. 

During examinations, you must do your own work. Talking or discussion is not permitted during the examinations, nor may you compare papers, copy from others, or collaborate in any way. Any collaborative behavior during the examinations will result in failure of the exam, and may lead to failure of the course and University disciplinary action.

Penalty for violation of this Code can also be extended to include failure of the course and University disciplinary action. 

\section{Week-wise Schedule}

Please consult the Live Syllabus for the updated schedule.

\end{document}

%%% Local Variables:
%%% mode: latex
%%% TeX-master: t
%%% End:
